\documentclass[journal,12pt,onecolumn]{IEEEtran}

% Import necessary packages
\usepackage{cite}
\usepackage{amsmath,amssymb,amsfonts,amsthm}
\usepackage{algorithmic}
\usepackage{graphicx}
\usepackage{textcomp}
\usepackage{xcolor}
\usepackage{txfonts}
\usepackage{listings}
\usepackage{enumitem}
\usepackage{mathtools}
\usepackage{gensymb}
\usepackage{comment}
\usepackage[breaklinks=true]{hyperref}
\usepackage{tkz-euclide} 
\usepackage{listings}
\usepackage{gvv}                                        
\usepackage[latin1]{inputenc}                                
\usepackage{color} 
\usepackage{enumitem} % Include this package
\usepackage{array}                                            
\usepackage{longtable}                                       
\usepackage{calc}                                             
\usepackage{multirow}                                         
\usepackage{hhline}                                           
\usepackage{ifthen}                                           
\usepackage{lscape}
\usepackage{tabularx}
\usepackage{array}
\usepackage{float}
\usepackage{multicol} % Add the multicol package
\usepackage{circuitikz}
\usepackage{lipsum}
\usepackage{textcomp}
% New theorem declarations
\newtheorem{theorem}{Theorem}[section]
\newtheorem{problem}{Problem}
\newtheorem{proposition}{Proposition}[section]
\newtheorem{lemma}{Lemma}[section]
\newtheorem{corollary}[theorem]{Corollary}
\newtheorem{example}{Example}[section]
\newtheorem{definition}[problem]{Definition}

% Custom command definitions
\newcommand{\BEQA}{\begin{eqnarray}}
\newcommand{\EEQA}{\end{eqnarray}}
\newcommand{\define}{\stackrel{\triangle}{=}}

\theoremstyle{remark}
\newtheorem{rem}{Remark}

% Document begins here
\begin{document}
\bibliographystyle{IEEEtran}
\vspace{3cm}

% Title of the document
\title{EIGENVALUES-HYBRID NEWTON POWER METHOD}
\author{EE24BTECH11011 - PRANAY}
\maketitle

\bigskip

% Custom figure and table numbering
\renewcommand{\thefigure}{\theenumi}
\renewcommand{\thetable}{\theenumi}
\subsection{\textbf{What are Eigenvalues?}}
An Eigenvalue is a scalar of linear operators for which there exists a non-zero vector. This property is equivalent to an Eigenvector.
Geometrically, Eigenvalue meaning is the transformation in the particular point of direction where it is stretched. In case, the Eigenvalue is negative, the direction gets reversed.\\


The eigenvalues of matrix are scalars by which some vectors (eigenvectors) change when the matrix (transformation) is applied to it. In other words, if $A$ is a square matrix of order $n \times n$ and v is a non-zero column vector of order $n \times 1$ such that $Av = \lambda$ (it means that the product of $A$ and $v$ is just a scalar multiple of v),then the scalar (real number) $\lambda$ is called an eigenvalue of the matrix $A$ that corresponds to the eigenvector $v$.
\subsection{\textbf{How to find?}}

To find the eigenvalues and eigenvectors of a square matrix $A$, we solve the eigenvalue equation:

$$Av = \lambda v$$

where:
* $A$ is an $n \times n$ matrix
* $v$ is a non-zero $n \times 1$ eigenvector
* $\lambda$ is an eigenvalue

We can rewrite this equation as:

$$(A - \lambda I)v = 0$$

where $I$ is the $n \times n$ identity matrix.

For a non-trivial solution to exist, the determinant of $(A - \lambda I)$ must be zero:

$$\det(A - \lambda I) = 0$$

This equation is called the **characteristic equation**. Solving this equation for $\lambda$ gives the eigenvalues.

Once we have the eigenvalues, we can find the corresponding eigenvectors by solving the system of linear equations $(A - \lambda I)v = 0$ for each eigenvalue $\lambda$.
\subsection{\textbf{HYBRID NEWTON POWER METHOD}}


The Hybrid Newton-Power method is a numerical technique to solve the eigenvalue problem $Ax = \lambda x$. 

\section{Hybrid Newton-Power Method}

The Hybrid Newton-Power method is a numerical technique to efficiently compute eigenvalues of large matrices. It combines the robustness of Newton's method with the simplicity of the power method.

\textbf{ALGORITHM}

\begin{enumerate}
    \item \textbf{Initialization:}
    \begin{align*}
        &\text{Choose an initial guess vector } \mathbf{v}^{(0)}. \\
        &\text{Set the iteration counter } k = 0.
    \end{align*}
    
    \item \textbf{Power Iteration:}
    \begin{align*}
        \mathbf{w}^{(k+1)} &= A\mathbf{v}^{(k)}, \\
        \mathbf{v}^{(k+1)} &= \frac{\mathbf{w}^{(k+1)}}{\|\mathbf{w}^{(k+1)}\|}.
    \end{align*}
    
    \item \textbf{Newton's Method Correction:}
    \begin{align*}
        \lambda^{(k)} &= \frac{\mathbf{v}^{(k)T}A\mathbf{v}^{(k)}}{\mathbf{v}^{(k)T}\mathbf{v}^{(k)}} \quad \text{(Rayleigh quotient)}, \\
        A^{(k)} &= A - \lambda^{(k)}I \quad \text{(shifted matrix)}, \\
        A^{(k)}\mathbf{u}^{(k)} &= \mathbf{v}^{(k)} \quad \text{(solve linear system)}, \\
        \mathbf{v}^{(k+1)} &= \mathbf{u}^{(k)}.
    \end{align*}
    
    \item \textbf{Convergence Check:}
    \begin{align*}
        &\text{If } \| \mathbf{v}^{(k+1)} - \mathbf{v}^{(k)} \| < \epsilon, \text{ stop.} \\
        &\text{Otherwise, increment } k \text{ and return to step 2.}
    \end{align*}
\end{enumerate}

\textbf{Convergence Properties:}
\begin{enumerate}
    \item The power iteration converges to the dominant eigenvalue.
    \item Newton's method refines the eigenvalue estimate and accelerates convergence.
    \item The method is particularly effective for large sparse matrices.
\end{enumerate}

\textbf{Implementation Considerations:}
\begin{enumerate}
    \item \textbf{Matrix Storage:} For large sparse matrices, consider using sparse matrix formats to reduce memory usage and computational cost.
    \item \textbf{Linear Solver:} The linear system in step 3 can be solved using direct or iterative methods.
    \item \textbf{Preconditioning:} Preconditioning techniques can improve the convergence rate of the linear solver.
    \item \textbf{Parallelism:} The power iteration and matrix-vector multiplication can be parallelized to speed up computations.
\end{enumerate}

By understanding the core concepts and implementation details, you can effectively apply the Hybrid Newton-Power method to solve a variety of eigenvalue problems in scientific and engineering applications.

\section{Advantages of the Hybrid Newton-Power Method}

The Hybrid Newton-Power method offers several advantages over traditional eigenvalue algorithms:

\begin{enumerate}
    \item \textbf{Robustness:}
	    
        \text{Combines the global convergence properties of the power method
        with the local quadratic convergence}
	    
    
    \item \textbf{Efficiency:}
    
        \text{Often converges faster especially for problems with multiple eigenvalues. Can handle large matrices.}
   
    
    \item \textbf{Simplicity:}
   
        \text{The power iteration step is simple to implement,making it accessible to a wide range of users.}

    
    \item \textbf{Flexibility:}
  
        \text{Can be adapted to find multiple eigenvalues using deflation techniques or shifting strategies.}
   
    
    \item \textbf{Accuracy:}
    
        \text{Newton's method correction ensures high accuracy in the computed eigenvalues and eigenvectors.}
    
\end{enumerate}


In conclusion, the Hybrid Newton-Power method provides a powerful and versatile tool for solving eigenvalue problems, particularly those involving large-scale matrices.
\section{Example: Hybrid Newton-Power Method}

Consider the matrix:
\begin{align}
\myvec{A} = 
\myvec{
4 & 1 \\
2 & 3
}.
\end{align}

We aim to find the dominant eigenvalue and corresponding eigenvector using the Hybrid Newton-Power method. 

\subsection*{Step-by-Step Solution}

\begin{enumerate}
    \item \textbf{Initialization:}
    \begin{align*}
        &\text{Choose an initial guess vector: } 
        \myvec{v}^{(0)} = \myvec{
        1 \\
        1
        }. \\
        &\text{Set } k = 0.
    \end{align*}
    
    \item \textbf{Power Iteration (Step 1):}
    \begin{align*}
        \myvec{w}^{(1)} &= \myvec{A}\myvec{v}^{(0)} = 
        \myvec{
        4 & 1 \\
        2 & 3
        }
        \myvec{
        1 \\
        1
        }
        =
        \myvec{
        5 \\
        5
        }, \\
        \myvec{v}^{(1)} &= \frac{\myvec{w}^{(1)}}{\|\myvec{w}^{(1)}\|} = 
        \frac{\myvec{
        5 \\
        5
        }}{\sqrt{5^2 + 5^2}} = 
        \myvec{
        \frac{1}{\sqrt{2}} \\
        \frac{1}{\sqrt{2}}
        }.
    \end{align*}
    
    \item \textbf{Newton's Method Correction:}
    \begin{align*}
        \lambda^{(0)} &= \frac{\myvec{v}^{(1)T}\myvec{A}\myvec{v}^{(1)}}{\myvec{v}^{(1)T}\myvec{v}^{(1)}} = 
        \frac{\myvec{
        \frac{1}{\sqrt{2}} & \frac{1}{\sqrt{2}}
        }
        \myvec{
        4 & 1 \\
        2 & 3
        }
        \myvec{
        \frac{1}{\sqrt{2}} \\
        \frac{1}{\sqrt{2}}
        }}{\myvec{
        \frac{1}{\sqrt{2}} & \frac{1}{\sqrt{2}}
        }
        \myvec{
        \frac{1}{\sqrt{2}} \\
        \frac{1}{\sqrt{2}}
        }}, \\
        &= \frac{\frac{5}{2}}{1} = 5.
    \end{align*}
    
    Form the shifted matrix:
    \begin{align*}
    \myvec{A}^{(0)} &= \myvec{A} - \lambda^{(0)}\myvec{I} = 
    \myvec{
    4 & 1 \\
    2 & 3
    } - 
    5 \myvec{
    1 & 0 \\
    0 & 1
    } = 
    \myvec{
    -1 & 1 \\
    2 & -2
    }.
    \end{align*}

    Solve the system:
    \begin{align*}
    \myvec{A}^{(0)}\myvec{u}^{(0)} &= \myvec{v}^{(1)}, \quad \text{so } 
    \myvec{
    -1 & 1 \\
    2 & -2
    }
    \myvec{u}^{(0)} = 
    \myvec{
    \frac{1}{\sqrt{2}} \\
    \frac{1}{\sqrt{2}}
    }.
    \end{align*}
    Solving gives:
    \begin{align*}
    \myvec{u}^{(0)} &= 
    \myvec{
    -\frac{1}{\sqrt{2}} \\
    0
    }.
    \end{align*}
    
    Update the eigenvector:
    \begin{align*}
    \myvec{v}^{(2)} &= \myvec{u}^{(0)}.
    \end{align*}

    \item \textbf{Convergence Check:}
    \begin{align*}
        &\text{If } \| \myvec{v}^{(2)} - \myvec{v}^{(1)} \| < \epsilon, \text{ stop. Otherwise, repeat.}
    \end{align*}
\end{enumerate}

The dominant eigenvalue approximated after iterations is \(\lambda \approx 5\), and the corresponding eigenvector is:
\begin{align*}
\myvec{v} &\approx 
\myvec{
\frac{1}{\sqrt{2}} \\
\frac{1}{\sqrt{2}}
}.
\end{align*}
\vspace*{\fill}  
\begin{center}  
    \fbox{\parbox{8cm}{\centering \textbf{Code at codes/code.c \\ if entry $5+4i$ enter it as $5$   $4$} \\ }}
\end{center}
\vspace*{\fill}
\end{document}

